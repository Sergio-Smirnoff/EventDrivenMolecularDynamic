\documentclass{beamer}
\usetheme{Warsaw}
\usepackage{graphicx}
\useoutertheme{miniframes}

% Datos
\title{Event Driven Molecular Dynamics}
\author{Grupo 5}
\institute{ITBA}
\date{} % sin fecha

% Numerar diapositivas
\setbeamertemplate{footline}[frame number]

\begin{document}

% Portada
\begin{frame}
  \titlepage
\end{frame}

% Introducción
\section{Introducción}

\subsection{Sistema Real}
\begin{frame}{Sistema Real}
  \begin{itemize}
    \item Breve descripción del sistema.
  \end{itemize}
\end{frame}

\subsection{Modelo Matemático}
\begin{frame}{Modelo Matemático}
  \begin{itemize}
    \item Ecuaciones generales.
  \end{itemize}
\end{frame}

% Implementación
\section{Implementación}
\begin{frame}{Implementación}
  \begin{itemize}
    \item Arquitectura del código.
    \item Pseudocódigo o UML.
  \end{itemize}
\end{frame}

% Simulaciones
\section{Simulaciones}
\begin{frame}{Simulaciones}
  \begin{itemize}
    \item Sistema particular simulado.
    \item Parámetros fijos y variables.
    \item Definición matemática de observables.
  \end{itemize}
\end{frame}

% Resultados
\section{Resultados}
\begin{frame}{Animaciones}
  % Aquí va la imagen de un frame representativo
  \includegraphics[width=0.7\linewidth]{ejemplo.png}
  
  \vspace{0.3cm}
  \footnotesize Link al video: \url{https://youtube.com/...}
\end{frame}

\begin{frame}{Evolución temporal del observable}
  \includegraphics[width=0.8\linewidth]{ejemplo_grafico.png}
  
  \vspace{0.3cm}
  Explicar el escalar característico (promedio, tasa, etc.).
\end{frame}

\begin{frame}{Input vs Observable}
  \includegraphics[width=0.8\linewidth]{ejemplo_resultados.png}
  
  \vspace{0.3cm}
  Mostrar promedios con barras de error.
\end{frame}

% Conclusiones
\section{Conclusiones}
\begin{frame}{Conclusiones}
  \begin{itemize}
    \item Conclusión 1 basada en los resultados.
    \item Conclusión 2…
  \end{itemize}
\end{frame}

% Cierre
\begin{frame}{}
  \centering
  \Huge ¡Gracias por su atención!
\end{frame}

\end{document}

